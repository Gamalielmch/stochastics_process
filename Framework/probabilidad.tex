\documentclass[11pt]{article}

    \usepackage[breakable]{tcolorbox}
    \usepackage{parskip} % Stop auto-indenting (to mimic markdown behaviour)
    
    \usepackage{iftex}
    \ifPDFTeX
    	\usepackage[T1]{fontenc}
    	\usepackage{mathpazo}
    \else
    	\usepackage{fontspec}
    \fi

    % Basic figure setup, for now with no caption control since it's done
    % automatically by Pandoc (which extracts ![](path) syntax from Markdown).
    \usepackage{graphicx}
    % Maintain compatibility with old templates. Remove in nbconvert 6.0
    \let\Oldincludegraphics\includegraphics
    % Ensure that by default, figures have no caption (until we provide a
    % proper Figure object with a Caption API and a way to capture that
    % in the conversion process - todo).
    \usepackage{caption}
    \DeclareCaptionFormat{nocaption}{}
    \captionsetup{format=nocaption,aboveskip=0pt,belowskip=0pt}

    \usepackage[Export]{adjustbox} % Used to constrain images to a maximum size
    \adjustboxset{max size={0.9\linewidth}{0.9\paperheight}}
    \usepackage{float}
    \floatplacement{figure}{H} % forces figures to be placed at the correct location
    \usepackage{xcolor} % Allow colors to be defined
    \usepackage{enumerate} % Needed for markdown enumerations to work
    \usepackage{geometry} % Used to adjust the document margins
    \usepackage{amsmath} % Equations
    \usepackage{amssymb} % Equations
    \usepackage{textcomp} % defines textquotesingle
    % Hack from http://tex.stackexchange.com/a/47451/13684:
    \AtBeginDocument{%
        \def\PYZsq{\textquotesingle}% Upright quotes in Pygmentized code
    }
    \usepackage{upquote} % Upright quotes for verbatim code
    \usepackage{eurosym} % defines \euro
    \usepackage[mathletters]{ucs} % Extended unicode (utf-8) support
    \usepackage{fancyvrb} % verbatim replacement that allows latex
    \usepackage{grffile} % extends the file name processing of package graphics 
                         % to support a larger range
    \makeatletter % fix for grffile with XeLaTeX
    \def\Gread@@xetex#1{%
      \IfFileExists{"\Gin@base".bb}%
      {\Gread@eps{\Gin@base.bb}}%
      {\Gread@@xetex@aux#1}%
    }
    \makeatother

    % The hyperref package gives us a pdf with properly built
    % internal navigation ('pdf bookmarks' for the table of contents,
    % internal cross-reference links, web links for URLs, etc.)
    \usepackage{hyperref}
    % The default LaTeX title has an obnoxious amount of whitespace. By default,
    % titling removes some of it. It also provides customization options.
    \usepackage{titling}
    \usepackage{longtable} % longtable support required by pandoc >1.10
    \usepackage{booktabs}  % table support for pandoc > 1.12.2
    \usepackage[inline]{enumitem} % IRkernel/repr support (it uses the enumerate* environment)
    \usepackage[normalem]{ulem} % ulem is needed to support strikethroughs (\sout)
                                % normalem makes italics be italics, not underlines
    \usepackage{mathrsfs}
    

    
    % Colors for the hyperref package
    \definecolor{urlcolor}{rgb}{0,.145,.698}
    \definecolor{linkcolor}{rgb}{.71,0.21,0.01}
    \definecolor{citecolor}{rgb}{.12,.54,.11}

    % ANSI colors
    \definecolor{ansi-black}{HTML}{3E424D}
    \definecolor{ansi-black-intense}{HTML}{282C36}
    \definecolor{ansi-red}{HTML}{E75C58}
    \definecolor{ansi-red-intense}{HTML}{B22B31}
    \definecolor{ansi-green}{HTML}{00A250}
    \definecolor{ansi-green-intense}{HTML}{007427}
    \definecolor{ansi-yellow}{HTML}{DDB62B}
    \definecolor{ansi-yellow-intense}{HTML}{B27D12}
    \definecolor{ansi-blue}{HTML}{208FFB}
    \definecolor{ansi-blue-intense}{HTML}{0065CA}
    \definecolor{ansi-magenta}{HTML}{D160C4}
    \definecolor{ansi-magenta-intense}{HTML}{A03196}
    \definecolor{ansi-cyan}{HTML}{60C6C8}
    \definecolor{ansi-cyan-intense}{HTML}{258F8F}
    \definecolor{ansi-white}{HTML}{C5C1B4}
    \definecolor{ansi-white-intense}{HTML}{A1A6B2}
    \definecolor{ansi-default-inverse-fg}{HTML}{FFFFFF}
    \definecolor{ansi-default-inverse-bg}{HTML}{000000}

    % commands and environments needed by pandoc snippets
    % extracted from the output of `pandoc -s`
    \providecommand{\tightlist}{%
      \setlength{\itemsep}{0pt}\setlength{\parskip}{0pt}}
    \DefineVerbatimEnvironment{Highlighting}{Verbatim}{commandchars=\\\{\}}
    % Add ',fontsize=\small' for more characters per line
    \newenvironment{Shaded}{}{}
    \newcommand{\KeywordTok}[1]{\textcolor[rgb]{0.00,0.44,0.13}{\textbf{{#1}}}}
    \newcommand{\DataTypeTok}[1]{\textcolor[rgb]{0.56,0.13,0.00}{{#1}}}
    \newcommand{\DecValTok}[1]{\textcolor[rgb]{0.25,0.63,0.44}{{#1}}}
    \newcommand{\BaseNTok}[1]{\textcolor[rgb]{0.25,0.63,0.44}{{#1}}}
    \newcommand{\FloatTok}[1]{\textcolor[rgb]{0.25,0.63,0.44}{{#1}}}
    \newcommand{\CharTok}[1]{\textcolor[rgb]{0.25,0.44,0.63}{{#1}}}
    \newcommand{\StringTok}[1]{\textcolor[rgb]{0.25,0.44,0.63}{{#1}}}
    \newcommand{\CommentTok}[1]{\textcolor[rgb]{0.38,0.63,0.69}{\textit{{#1}}}}
    \newcommand{\OtherTok}[1]{\textcolor[rgb]{0.00,0.44,0.13}{{#1}}}
    \newcommand{\AlertTok}[1]{\textcolor[rgb]{1.00,0.00,0.00}{\textbf{{#1}}}}
    \newcommand{\FunctionTok}[1]{\textcolor[rgb]{0.02,0.16,0.49}{{#1}}}
    \newcommand{\RegionMarkerTok}[1]{{#1}}
    \newcommand{\ErrorTok}[1]{\textcolor[rgb]{1.00,0.00,0.00}{\textbf{{#1}}}}
    \newcommand{\NormalTok}[1]{{#1}}
    
    % Additional commands for more recent versions of Pandoc
    \newcommand{\ConstantTok}[1]{\textcolor[rgb]{0.53,0.00,0.00}{{#1}}}
    \newcommand{\SpecialCharTok}[1]{\textcolor[rgb]{0.25,0.44,0.63}{{#1}}}
    \newcommand{\VerbatimStringTok}[1]{\textcolor[rgb]{0.25,0.44,0.63}{{#1}}}
    \newcommand{\SpecialStringTok}[1]{\textcolor[rgb]{0.73,0.40,0.53}{{#1}}}
    \newcommand{\ImportTok}[1]{{#1}}
    \newcommand{\DocumentationTok}[1]{\textcolor[rgb]{0.73,0.13,0.13}{\textit{{#1}}}}
    \newcommand{\AnnotationTok}[1]{\textcolor[rgb]{0.38,0.63,0.69}{\textbf{\textit{{#1}}}}}
    \newcommand{\CommentVarTok}[1]{\textcolor[rgb]{0.38,0.63,0.69}{\textbf{\textit{{#1}}}}}
    \newcommand{\VariableTok}[1]{\textcolor[rgb]{0.10,0.09,0.49}{{#1}}}
    \newcommand{\ControlFlowTok}[1]{\textcolor[rgb]{0.00,0.44,0.13}{\textbf{{#1}}}}
    \newcommand{\OperatorTok}[1]{\textcolor[rgb]{0.40,0.40,0.40}{{#1}}}
    \newcommand{\BuiltInTok}[1]{{#1}}
    \newcommand{\ExtensionTok}[1]{{#1}}
    \newcommand{\PreprocessorTok}[1]{\textcolor[rgb]{0.74,0.48,0.00}{{#1}}}
    \newcommand{\AttributeTok}[1]{\textcolor[rgb]{0.49,0.56,0.16}{{#1}}}
    \newcommand{\InformationTok}[1]{\textcolor[rgb]{0.38,0.63,0.69}{\textbf{\textit{{#1}}}}}
    \newcommand{\WarningTok}[1]{\textcolor[rgb]{0.38,0.63,0.69}{\textbf{\textit{{#1}}}}}
    
    
    % Define a nice break command that doesn't care if a line doesn't already
    % exist.
    \def\br{\hspace*{\fill} \\* }
    % Math Jax compatibility definitions
    \def\gt{>}
    \def\lt{<}
    \let\Oldtex\TeX
    \let\Oldlatex\LaTeX
    \renewcommand{\TeX}{\textrm{\Oldtex}}
    \renewcommand{\LaTeX}{\textrm{\Oldlatex}}
    % Document parameters
    % Document title
    \title{Probabilidad}
    
    
    
    
    
% Pygments definitions
\makeatletter
\def\PY@reset{\let\PY@it=\relax \let\PY@bf=\relax%
    \let\PY@ul=\relax \let\PY@tc=\relax%
    \let\PY@bc=\relax \let\PY@ff=\relax}
\def\PY@tok#1{\csname PY@tok@#1\endcsname}
\def\PY@toks#1+{\ifx\relax#1\empty\else%
    \PY@tok{#1}\expandafter\PY@toks\fi}
\def\PY@do#1{\PY@bc{\PY@tc{\PY@ul{%
    \PY@it{\PY@bf{\PY@ff{#1}}}}}}}
\def\PY#1#2{\PY@reset\PY@toks#1+\relax+\PY@do{#2}}

\expandafter\def\csname PY@tok@w\endcsname{\def\PY@tc##1{\textcolor[rgb]{0.73,0.73,0.73}{##1}}}
\expandafter\def\csname PY@tok@c\endcsname{\let\PY@it=\textit\def\PY@tc##1{\textcolor[rgb]{0.25,0.50,0.50}{##1}}}
\expandafter\def\csname PY@tok@cp\endcsname{\def\PY@tc##1{\textcolor[rgb]{0.74,0.48,0.00}{##1}}}
\expandafter\def\csname PY@tok@k\endcsname{\let\PY@bf=\textbf\def\PY@tc##1{\textcolor[rgb]{0.00,0.50,0.00}{##1}}}
\expandafter\def\csname PY@tok@kp\endcsname{\def\PY@tc##1{\textcolor[rgb]{0.00,0.50,0.00}{##1}}}
\expandafter\def\csname PY@tok@kt\endcsname{\def\PY@tc##1{\textcolor[rgb]{0.69,0.00,0.25}{##1}}}
\expandafter\def\csname PY@tok@o\endcsname{\def\PY@tc##1{\textcolor[rgb]{0.40,0.40,0.40}{##1}}}
\expandafter\def\csname PY@tok@ow\endcsname{\let\PY@bf=\textbf\def\PY@tc##1{\textcolor[rgb]{0.67,0.13,1.00}{##1}}}
\expandafter\def\csname PY@tok@nb\endcsname{\def\PY@tc##1{\textcolor[rgb]{0.00,0.50,0.00}{##1}}}
\expandafter\def\csname PY@tok@nf\endcsname{\def\PY@tc##1{\textcolor[rgb]{0.00,0.00,1.00}{##1}}}
\expandafter\def\csname PY@tok@nc\endcsname{\let\PY@bf=\textbf\def\PY@tc##1{\textcolor[rgb]{0.00,0.00,1.00}{##1}}}
\expandafter\def\csname PY@tok@nn\endcsname{\let\PY@bf=\textbf\def\PY@tc##1{\textcolor[rgb]{0.00,0.00,1.00}{##1}}}
\expandafter\def\csname PY@tok@ne\endcsname{\let\PY@bf=\textbf\def\PY@tc##1{\textcolor[rgb]{0.82,0.25,0.23}{##1}}}
\expandafter\def\csname PY@tok@nv\endcsname{\def\PY@tc##1{\textcolor[rgb]{0.10,0.09,0.49}{##1}}}
\expandafter\def\csname PY@tok@no\endcsname{\def\PY@tc##1{\textcolor[rgb]{0.53,0.00,0.00}{##1}}}
\expandafter\def\csname PY@tok@nl\endcsname{\def\PY@tc##1{\textcolor[rgb]{0.63,0.63,0.00}{##1}}}
\expandafter\def\csname PY@tok@ni\endcsname{\let\PY@bf=\textbf\def\PY@tc##1{\textcolor[rgb]{0.60,0.60,0.60}{##1}}}
\expandafter\def\csname PY@tok@na\endcsname{\def\PY@tc##1{\textcolor[rgb]{0.49,0.56,0.16}{##1}}}
\expandafter\def\csname PY@tok@nt\endcsname{\let\PY@bf=\textbf\def\PY@tc##1{\textcolor[rgb]{0.00,0.50,0.00}{##1}}}
\expandafter\def\csname PY@tok@nd\endcsname{\def\PY@tc##1{\textcolor[rgb]{0.67,0.13,1.00}{##1}}}
\expandafter\def\csname PY@tok@s\endcsname{\def\PY@tc##1{\textcolor[rgb]{0.73,0.13,0.13}{##1}}}
\expandafter\def\csname PY@tok@sd\endcsname{\let\PY@it=\textit\def\PY@tc##1{\textcolor[rgb]{0.73,0.13,0.13}{##1}}}
\expandafter\def\csname PY@tok@si\endcsname{\let\PY@bf=\textbf\def\PY@tc##1{\textcolor[rgb]{0.73,0.40,0.53}{##1}}}
\expandafter\def\csname PY@tok@se\endcsname{\let\PY@bf=\textbf\def\PY@tc##1{\textcolor[rgb]{0.73,0.40,0.13}{##1}}}
\expandafter\def\csname PY@tok@sr\endcsname{\def\PY@tc##1{\textcolor[rgb]{0.73,0.40,0.53}{##1}}}
\expandafter\def\csname PY@tok@ss\endcsname{\def\PY@tc##1{\textcolor[rgb]{0.10,0.09,0.49}{##1}}}
\expandafter\def\csname PY@tok@sx\endcsname{\def\PY@tc##1{\textcolor[rgb]{0.00,0.50,0.00}{##1}}}
\expandafter\def\csname PY@tok@m\endcsname{\def\PY@tc##1{\textcolor[rgb]{0.40,0.40,0.40}{##1}}}
\expandafter\def\csname PY@tok@gh\endcsname{\let\PY@bf=\textbf\def\PY@tc##1{\textcolor[rgb]{0.00,0.00,0.50}{##1}}}
\expandafter\def\csname PY@tok@gu\endcsname{\let\PY@bf=\textbf\def\PY@tc##1{\textcolor[rgb]{0.50,0.00,0.50}{##1}}}
\expandafter\def\csname PY@tok@gd\endcsname{\def\PY@tc##1{\textcolor[rgb]{0.63,0.00,0.00}{##1}}}
\expandafter\def\csname PY@tok@gi\endcsname{\def\PY@tc##1{\textcolor[rgb]{0.00,0.63,0.00}{##1}}}
\expandafter\def\csname PY@tok@gr\endcsname{\def\PY@tc##1{\textcolor[rgb]{1.00,0.00,0.00}{##1}}}
\expandafter\def\csname PY@tok@ge\endcsname{\let\PY@it=\textit}
\expandafter\def\csname PY@tok@gs\endcsname{\let\PY@bf=\textbf}
\expandafter\def\csname PY@tok@gp\endcsname{\let\PY@bf=\textbf\def\PY@tc##1{\textcolor[rgb]{0.00,0.00,0.50}{##1}}}
\expandafter\def\csname PY@tok@go\endcsname{\def\PY@tc##1{\textcolor[rgb]{0.53,0.53,0.53}{##1}}}
\expandafter\def\csname PY@tok@gt\endcsname{\def\PY@tc##1{\textcolor[rgb]{0.00,0.27,0.87}{##1}}}
\expandafter\def\csname PY@tok@err\endcsname{\def\PY@bc##1{\setlength{\fboxsep}{0pt}\fcolorbox[rgb]{1.00,0.00,0.00}{1,1,1}{\strut ##1}}}
\expandafter\def\csname PY@tok@kc\endcsname{\let\PY@bf=\textbf\def\PY@tc##1{\textcolor[rgb]{0.00,0.50,0.00}{##1}}}
\expandafter\def\csname PY@tok@kd\endcsname{\let\PY@bf=\textbf\def\PY@tc##1{\textcolor[rgb]{0.00,0.50,0.00}{##1}}}
\expandafter\def\csname PY@tok@kn\endcsname{\let\PY@bf=\textbf\def\PY@tc##1{\textcolor[rgb]{0.00,0.50,0.00}{##1}}}
\expandafter\def\csname PY@tok@kr\endcsname{\let\PY@bf=\textbf\def\PY@tc##1{\textcolor[rgb]{0.00,0.50,0.00}{##1}}}
\expandafter\def\csname PY@tok@bp\endcsname{\def\PY@tc##1{\textcolor[rgb]{0.00,0.50,0.00}{##1}}}
\expandafter\def\csname PY@tok@fm\endcsname{\def\PY@tc##1{\textcolor[rgb]{0.00,0.00,1.00}{##1}}}
\expandafter\def\csname PY@tok@vc\endcsname{\def\PY@tc##1{\textcolor[rgb]{0.10,0.09,0.49}{##1}}}
\expandafter\def\csname PY@tok@vg\endcsname{\def\PY@tc##1{\textcolor[rgb]{0.10,0.09,0.49}{##1}}}
\expandafter\def\csname PY@tok@vi\endcsname{\def\PY@tc##1{\textcolor[rgb]{0.10,0.09,0.49}{##1}}}
\expandafter\def\csname PY@tok@vm\endcsname{\def\PY@tc##1{\textcolor[rgb]{0.10,0.09,0.49}{##1}}}
\expandafter\def\csname PY@tok@sa\endcsname{\def\PY@tc##1{\textcolor[rgb]{0.73,0.13,0.13}{##1}}}
\expandafter\def\csname PY@tok@sb\endcsname{\def\PY@tc##1{\textcolor[rgb]{0.73,0.13,0.13}{##1}}}
\expandafter\def\csname PY@tok@sc\endcsname{\def\PY@tc##1{\textcolor[rgb]{0.73,0.13,0.13}{##1}}}
\expandafter\def\csname PY@tok@dl\endcsname{\def\PY@tc##1{\textcolor[rgb]{0.73,0.13,0.13}{##1}}}
\expandafter\def\csname PY@tok@s2\endcsname{\def\PY@tc##1{\textcolor[rgb]{0.73,0.13,0.13}{##1}}}
\expandafter\def\csname PY@tok@sh\endcsname{\def\PY@tc##1{\textcolor[rgb]{0.73,0.13,0.13}{##1}}}
\expandafter\def\csname PY@tok@s1\endcsname{\def\PY@tc##1{\textcolor[rgb]{0.73,0.13,0.13}{##1}}}
\expandafter\def\csname PY@tok@mb\endcsname{\def\PY@tc##1{\textcolor[rgb]{0.40,0.40,0.40}{##1}}}
\expandafter\def\csname PY@tok@mf\endcsname{\def\PY@tc##1{\textcolor[rgb]{0.40,0.40,0.40}{##1}}}
\expandafter\def\csname PY@tok@mh\endcsname{\def\PY@tc##1{\textcolor[rgb]{0.40,0.40,0.40}{##1}}}
\expandafter\def\csname PY@tok@mi\endcsname{\def\PY@tc##1{\textcolor[rgb]{0.40,0.40,0.40}{##1}}}
\expandafter\def\csname PY@tok@il\endcsname{\def\PY@tc##1{\textcolor[rgb]{0.40,0.40,0.40}{##1}}}
\expandafter\def\csname PY@tok@mo\endcsname{\def\PY@tc##1{\textcolor[rgb]{0.40,0.40,0.40}{##1}}}
\expandafter\def\csname PY@tok@ch\endcsname{\let\PY@it=\textit\def\PY@tc##1{\textcolor[rgb]{0.25,0.50,0.50}{##1}}}
\expandafter\def\csname PY@tok@cm\endcsname{\let\PY@it=\textit\def\PY@tc##1{\textcolor[rgb]{0.25,0.50,0.50}{##1}}}
\expandafter\def\csname PY@tok@cpf\endcsname{\let\PY@it=\textit\def\PY@tc##1{\textcolor[rgb]{0.25,0.50,0.50}{##1}}}
\expandafter\def\csname PY@tok@c1\endcsname{\let\PY@it=\textit\def\PY@tc##1{\textcolor[rgb]{0.25,0.50,0.50}{##1}}}
\expandafter\def\csname PY@tok@cs\endcsname{\let\PY@it=\textit\def\PY@tc##1{\textcolor[rgb]{0.25,0.50,0.50}{##1}}}

\def\PYZbs{\char`\\}
\def\PYZus{\char`\_}
\def\PYZob{\char`\{}
\def\PYZcb{\char`\}}
\def\PYZca{\char`\^}
\def\PYZam{\char`\&}
\def\PYZlt{\char`\<}
\def\PYZgt{\char`\>}
\def\PYZsh{\char`\#}
\def\PYZpc{\char`\%}
\def\PYZdl{\char`\$}
\def\PYZhy{\char`\-}
\def\PYZsq{\char`\'}
\def\PYZdq{\char`\"}
\def\PYZti{\char`\~}
% for compatibility with earlier versions
\def\PYZat{@}
\def\PYZlb{[}
\def\PYZrb{]}
\makeatother


    % For linebreaks inside Verbatim environment from package fancyvrb. 
    \makeatletter
        \newbox\Wrappedcontinuationbox 
        \newbox\Wrappedvisiblespacebox 
        \newcommand*\Wrappedvisiblespace {\textcolor{red}{\textvisiblespace}} 
        \newcommand*\Wrappedcontinuationsymbol {\textcolor{red}{\llap{\tiny$\m@th\hookrightarrow$}}} 
        \newcommand*\Wrappedcontinuationindent {3ex } 
        \newcommand*\Wrappedafterbreak {\kern\Wrappedcontinuationindent\copy\Wrappedcontinuationbox} 
        % Take advantage of the already applied Pygments mark-up to insert 
        % potential linebreaks for TeX processing. 
        %        {, <, #, %, $, ' and ": go to next line. 
        %        _, }, ^, &, >, - and ~: stay at end of broken line. 
        % Use of \textquotesingle for straight quote. 
        \newcommand*\Wrappedbreaksatspecials {% 
            \def\PYGZus{\discretionary{\char`\_}{\Wrappedafterbreak}{\char`\_}}% 
            \def\PYGZob{\discretionary{}{\Wrappedafterbreak\char`\{}{\char`\{}}% 
            \def\PYGZcb{\discretionary{\char`\}}{\Wrappedafterbreak}{\char`\}}}% 
            \def\PYGZca{\discretionary{\char`\^}{\Wrappedafterbreak}{\char`\^}}% 
            \def\PYGZam{\discretionary{\char`\&}{\Wrappedafterbreak}{\char`\&}}% 
            \def\PYGZlt{\discretionary{}{\Wrappedafterbreak\char`\<}{\char`\<}}% 
            \def\PYGZgt{\discretionary{\char`\>}{\Wrappedafterbreak}{\char`\>}}% 
            \def\PYGZsh{\discretionary{}{\Wrappedafterbreak\char`\#}{\char`\#}}% 
            \def\PYGZpc{\discretionary{}{\Wrappedafterbreak\char`\%}{\char`\%}}% 
            \def\PYGZdl{\discretionary{}{\Wrappedafterbreak\char`\$}{\char`\$}}% 
            \def\PYGZhy{\discretionary{\char`\-}{\Wrappedafterbreak}{\char`\-}}% 
            \def\PYGZsq{\discretionary{}{\Wrappedafterbreak\textquotesingle}{\textquotesingle}}% 
            \def\PYGZdq{\discretionary{}{\Wrappedafterbreak\char`\"}{\char`\"}}% 
            \def\PYGZti{\discretionary{\char`\~}{\Wrappedafterbreak}{\char`\~}}% 
        } 
        % Some characters . , ; ? ! / are not pygmentized. 
        % This macro makes them "active" and they will insert potential linebreaks 
        \newcommand*\Wrappedbreaksatpunct {% 
            \lccode`\~`\.\lowercase{\def~}{\discretionary{\hbox{\char`\.}}{\Wrappedafterbreak}{\hbox{\char`\.}}}% 
            \lccode`\~`\,\lowercase{\def~}{\discretionary{\hbox{\char`\,}}{\Wrappedafterbreak}{\hbox{\char`\,}}}% 
            \lccode`\~`\;\lowercase{\def~}{\discretionary{\hbox{\char`\;}}{\Wrappedafterbreak}{\hbox{\char`\;}}}% 
            \lccode`\~`\:\lowercase{\def~}{\discretionary{\hbox{\char`\:}}{\Wrappedafterbreak}{\hbox{\char`\:}}}% 
            \lccode`\~`\?\lowercase{\def~}{\discretionary{\hbox{\char`\?}}{\Wrappedafterbreak}{\hbox{\char`\?}}}% 
            \lccode`\~`\!\lowercase{\def~}{\discretionary{\hbox{\char`\!}}{\Wrappedafterbreak}{\hbox{\char`\!}}}% 
            \lccode`\~`\/\lowercase{\def~}{\discretionary{\hbox{\char`\/}}{\Wrappedafterbreak}{\hbox{\char`\/}}}% 
            \catcode`\.\active
            \catcode`\,\active 
            \catcode`\;\active
            \catcode`\:\active
            \catcode`\?\active
            \catcode`\!\active
            \catcode`\/\active 
            \lccode`\~`\~ 	
        }
    \makeatother

    \let\OriginalVerbatim=\Verbatim
    \makeatletter
    \renewcommand{\Verbatim}[1][1]{%
        %\parskip\z@skip
        \sbox\Wrappedcontinuationbox {\Wrappedcontinuationsymbol}%
        \sbox\Wrappedvisiblespacebox {\FV@SetupFont\Wrappedvisiblespace}%
        \def\FancyVerbFormatLine ##1{\hsize\linewidth
            \vtop{\raggedright\hyphenpenalty\z@\exhyphenpenalty\z@
                \doublehyphendemerits\z@\finalhyphendemerits\z@
                \strut ##1\strut}%
        }%
        % If the linebreak is at a space, the latter will be displayed as visible
        % space at end of first line, and a continuation symbol starts next line.
        % Stretch/shrink are however usually zero for typewriter font.
        \def\FV@Space {%
            \nobreak\hskip\z@ plus\fontdimen3\font minus\fontdimen4\font
            \discretionary{\copy\Wrappedvisiblespacebox}{\Wrappedafterbreak}
            {\kern\fontdimen2\font}%
        }%
        
        % Allow breaks at special characters using \PYG... macros.
        \Wrappedbreaksatspecials
        % Breaks at punctuation characters . , ; ? ! and / need catcode=\active 	
        \OriginalVerbatim[#1,codes*=\Wrappedbreaksatpunct]%
    }
    \makeatother

    % Exact colors from NB
    \definecolor{incolor}{HTML}{303F9F}
    \definecolor{outcolor}{HTML}{D84315}
    \definecolor{cellborder}{HTML}{CFCFCF}
    \definecolor{cellbackground}{HTML}{F7F7F7}
    
    % prompt
    \makeatletter
    \newcommand{\boxspacing}{\kern\kvtcb@left@rule\kern\kvtcb@boxsep}
    \makeatother
    \newcommand{\prompt}[4]{
        \ttfamily\llap{{\color{#2}[#3]:\hspace{3pt}#4}}\vspace{-\baselineskip}
    }
    

    
    % Prevent overflowing lines due to hard-to-break entities
    \sloppy 
    % Setup hyperref package
    \hypersetup{
      breaklinks=true,  % so long urls are correctly broken across lines
      colorlinks=true,
      urlcolor=urlcolor,
      linkcolor=linkcolor,
      citecolor=citecolor,
      }
    % Slightly bigger margins than the latex defaults
    
    \geometry{verbose,tmargin=1in,bmargin=1in,lmargin=1in,rmargin=1in}
    
    

\begin{document}
    
    \maketitle
    
    

    
    \hypertarget{probabilidad}{%
\section{Probabilidad}\label{probabilidad}}

    \hypertarget{definiciones-buxe1sicas}{%
\subsection{Definiciones básicas}\label{definiciones-buxe1sicas}}

En experimentos aleatorios, la lista de todos los posibles resultados se
denomia \textbf{espacio muestral}, denotado por \(S\)

\begin{itemize}
\item
  La lista consiste de \textbf{resultados indiviuduales} o
  \textbf{elementos}. El esapcio muestral puede ser finito o infinito y
  puede ser discreto o continuo. Estos elementos tienen las propiedades
  de que son \emph{mutuamente excluyentes} y de que son
  \emph{exhaustivos}.
\item
  Un evento es un subconjunto de un espacio muestral \(S\): estos
  generalmente se denotan con letras mayúsculas, \(A, B,\) etc.
  Denotamos por \(P(A)\) la probabilidad de que el evento \(A\) ocurra
  en cada repetición del experimento aleatorio.
\end{itemize}

\emph{Ejercicios} - Considere el experimento de lanzar un dado. Escriba
el espacio muestral si nos interesara el número que aparece en la cara
superior - Considere el experimento de lanzar un dado. Escriba el
espacio muestral si nos interesara si es par o impar - Escriba el
espacio muestral de las figuras anteriores. - Denote el conjunto de
ciudades con mas de un millon de habitantes - Denote el conjunto de
todos los puntos \((x,y)\) que se encuentran dentro de un circulos de
diametro 4 con centro en el origen - Dado el espacio muestral \$S =
\lbrace t \vert t \leq 0 \rbrace \$ , donde \(t\) es la vida en años de
cierto componen te electrónico. Denote el evento \(A\) de que el
componente falle antes de que finalice el quinto año

    \hypertarget{conjuntos}{%
\subsection{Conjuntos}\label{conjuntos}}

Es conveniente utilizar la notación de \textbf{conjuntos} al derivar
probabilidades de eventos. Esto lleva a que el espacio muestral \(S\) se
denomine \emph{conjunto universal}, el conjunto de todos los resultados:
un evento \(A\) es un \emph{subconjunto} de \(S\). Estos nos permite
denotar operaciones con conjuntos.

\begin{itemize}
\item
  Union
  \(A \cup B = \lbrace x\vert x \in A \text{ o } x \in B \text{ o ambos }\rbrace\)
\item
  Intersección
  \(A \cap B = \lbrace x\vert x \in A \text{ y } x \in B \rbrace\)
\item
  Complement \(A^{c}= \lbrace x \vert x \notin A \rbrace\)
\end{itemize}

¿Los siguientes diagramas que operaciones representan?

\begin{itemize}
\tightlist
\item
  El \textbf{conjunto vacio} se denonta como \(\phi\) este conjunto no
  tiene elementos. Dos ejemplos de conjuntos vacios \(S^{c}=\phi\) y si
  \(A\) y \(B\) \textbf{son conjuntos mutamente excluyentes} es decir
  que no tiene elementos en común, entonces \(A \cap B= \phi\), se dice
  que son conjuntos disjuntos.
\end{itemize}

\emph{Ejercicios} - Sean
\(V = \lbrace a , e, i, o, u \rbrace y C = \lbrace l, r, s, t\rbrace\);
encuentre \(V \cap C\) - Sean \$A = \lbrace a, b, c \rbrace \$ y \$B =
\lbrace b, c, d, e\rbrace \$; encuentre \(A \cap B\) - Si
\(M = \lbrace x \vert 3 < x < 9 \rbrace y N = \lbrace y \vert 5 < y < 12\rbrace\),
encuentre \(M \cup N\)\\
- Del siguiente diagrama encuentre las regiones en (a)\(A \cup C\),
(b)\(B´ \cap C\), (c)\(A\cap B \cap C\), (d) \((A \cup B) \cap C'\)

    \hypertarget{estimaciuxf3n-de-probabilidad-de-un-evento}{%
\subsection{Estimación de probabilidad de un
evento}\label{estimaciuxf3n-de-probabilidad-de-un-evento}}

Conteo del número de puntos.

Si una operación se puede llevar a cabo en \(n_{1}\) formas, y si para
cada una de éstas se puede realizar una segunda operación en \(n_{2}\)
formas, entonces las dos operaciones se pueden ejecutar juntas de
\(n_{1}n_{2}\) formas.

Existen dos forma para estimar la probabilidad de un evento \(A\), la
primera es la frecuencia relativa:

\[ \frac{\text{la cantidad de veces que ocurre un evento particular } A}{\text{número total de ensayos}}\]

La otra es que si se tiene un conjunto finito y los resultados son
equiprobables, como el lanzamiento de un dado, entonces la probabilidad
es estimado por

\[ P(A)= \frac{\text{número de elementos de $S$ cuando $A$ ocurre}}{\text{número de elementos en $S$}}\]

\hypertarget{probabilidad-de-eventos-y-conjuntos}{%
\subsection{Probabilidad de eventos y
conjuntos}\label{probabilidad-de-eventos-y-conjuntos}}

La probabilidad de cualquier evento debe satisfacer los siguientes tres
axiomas

\begin{itemize}
\tightlist
\item
  Axioma 1: \(0 \leq P(A) \leq 1\) para cada evento \(A\)
\item
  Axioma 2: \$P(S)=1 \$
\item
  Axioma 3: \(P(A\cup B)=P(A)+P(B)\) si y sólo si \(A\) y \(B\) son
  mutamente excluyentes (\(A\cap B=\phi\))
\item
  Axioma 4: \(P(\phi)=0\)
\end{itemize}

Estos cuatro axiomas se pueden extender a más de eventos mutamente
excluyentes:

En \(S\) donde
\(A_{i} \cap A_{j} = \phi \text{  } \forall \text{  } i\neq j\). Esto es
conocido como partición de \(S\) si

\begin{itemize}
\tightlist
\item
  \(A_{i} \cap A_{j} = \phi \text{  } \forall \text{  } i\neq j\).
\item
  \(\bigcup\limits_{i=1}^{k} A_{i}= A_{1} \cup A_{2} \cup \ldots \cup A_{3}= S\).
  \(A_{1},A_{2}, \ldots ,A_{k}\) es una \textbf{lista exhaustiva} tal
  que one de los eventos debe ocurrir
\item
  \(P(A_{i})>0\)
\end{itemize}

\hypertarget{teoremas}{%
\subsubsection{Teoremas}\label{teoremas}}

\begin{itemize}
\item
  \(P(S) = P(A_{1}\cup A_{2} \cup \cdots \cup A_{k}) \displaystyle \sum_{i=1}^{k}{P(A_{i})}=1\)
\item
  \(P(A^{c})=1-P(A)\)
\item
  \(P(A\cup B)= P(A)+P(B)-P(A\cap B)\)
\item
  \(P(S)=P(A \cup A^{c})= P(A)+P(A^{c})=1\)
\item
  Para cualquier conjunto \(A\) y \(B\)\\
  \(A\cup B= A\cup (B\cap A^{c})\), \(B=(A\cap B)\cup (B\cap A^{c})\)
\end{itemize}

en donde \(A\) y \(B \cap A^{c}\) son conjuntos disjuntos y \(A\cap B\)
and \(B\cap A^{c}\) son conjuntos disjuntos. Por lo tanto

\begin{itemize}
\tightlist
\item
  \$P(A\cup B)=P(A)+P(B\cap A\^{}\{c\}) \$
\item
  \$P(B)=P(A\cup B)+P(B\cap A\^{}\{c\}) \$
\end{itemize}

Estas dos expresiones nos llevan a - \(P(A\cup B)=P(A)+P(B)-P(A\cup B)\)

\emph{Ejemplos}

\begin{itemize}
\item
  Una moneda se lanza dos veces. ¿Cuál es la probabilidad de que ocurra
  al menos una cara (H)?
\item
  Se tiran dos dados honestos distinguibles a y b y se anotan los
  valores de las caras superiores. ¿Cuáles son los elementos del espacio
  muestral? ¿Cuál es la probabilidad de que la suma de los valores de
  los dos dados sea 7? ¿Cuál es la probabilidad de que aparezca al menos
  un 5?
\item
  Se carga un dado de forma que exista el doble de probabilidades de que
  salga un número par que uno impar. Si E es el evento de que ocurra un
  número menor que 4 en un solo lanzamiento del dado, calcule \(P(E)\).
\item
  Del ejemplo anterior sea \(A\) el evento de que resulte un número par
  y sea \(B\) el evento de que resulte un número divisible entre 3.
  Calcule \(P(A \cup B)\) y \(P(A \cap B)\).
\item
  De un mazo bien mezclado de 52 naipes, se saca una sola carta al azar.
  Calcula la probabilidad de que sea un corazón o un as.
\item
  Suponga tres eventos \(A, B\) y \(C\), que no son excluyentes
  encuentre \(P(A \cup B \cup C)\)
\end{itemize}

    \hypertarget{probabilidad-condicional-e-independencia}{%
\subsection{Probabilidad condicional e
independencia}\label{probabilidad-condicional-e-independencia}}

Si la ocurrencia de un evento \(B\) se ve afectada por la ocurrencia de
otro evento \(A\), entonces decimos que \(A\) y \(B\) son eventos
dependientes. Cuando se realiza el experimento, se sabe que ha ocurrido
el evento \(A\). ¿Afecta esto la probabilidad de \(B\)? Sí lo hace se
convierte en una probabilidad condicional de \(B\) dada \(A\), escrita
como \(P (B \vert A)\). Por lo general, esto será distinto de la
probabilidad \(P (B)\). La probabilidad condicional de B está
restringida a la parte del espacio muestral donde \(A\) ocurrió. Esta
probabilidad condicional se define como

\[
P(B\vert A)= \frac{P(A\cap B)}{ P(A)} \text{,        } \quad P(A)>0
\]

o su equivalente

\[
P(A\cap B)= P(A) P(B\vert A)  \text{,        } \quad P(A)>0
\]

En términos de conteo, suponga que un experimento se repite \(N\) veces,
de las cuales \(A\) ocurre \(N (A)\) veces, y \(A\) dado por \(B\)
ocurre \(N (B ∩ A)\) veces. La proporción de veces que ocurre \(B\) es

\[
\frac{N(A\cap B)}{N(A)}= \frac{N(A\cap B)}{N}\frac{ N}{N(A)}
\]

Si la probabilidad de \(B\) no es afectada por la ocurrencia de \(A\),
entonces son \emph{eventos independientes} \[
P(B\vert A)=P(B)
\]

lo que implica que \[
P(A\cap B)=P(A)P(B)
\]

    \hypertarget{ejemplos}{%
\subsubsection{\texorpdfstring{\emph{Ejemplos}}{Ejemplos}}\label{ejemplos}}

\emph{Ejercicios} - suponga que tenemos un espacio muestral S
constituido por la población de adultos de una pequeña ciudad

\begin{longtable}[]{@{}cccc@{}}
\toprule
& Empleado & Desempleado & Total\tabularnewline
\midrule
\endhead
Hombre & 460 & 40 & 500\tabularnewline
Mujer & 140 & 260 & 400\tabularnewline
Total & 600 & 300 & 900\tabularnewline
\bottomrule
\end{longtable}

Se seleccionará al azar a uno de estos individuos ¿Cuál es la
probabilidad de que sea hombre dado que tiene empleo?

\begin{itemize}
\item
  Suponga que tenemos una caja de fusibles que contiene 20 unidades, de
  las cuales 5 están defectuosas. Si se seleccionan 2 fusibles al azar y
  se retiran de la caja, uno después del otro, sin reemplazar el
  primero, ¿cuál es la probabilidad de que ambos fusibles estén
  defectuosos?
\item
  Considere el evento \(B\) de obtener un cuadrado perfecto cuando se
  lanza un dado. El dado se construye de modo que los números pares
  tengan el doble de probabilidad de ocurrencia que los números nones.
  Con base en el espacio muestral S = \{1, 2, 3, 4, 5, 6\}, ¿Cuál es la
  probabilidad de que ocurra \(B\)? Suponga ahora que se sabe que el
  lanzamiento del dado tiene como resultado un número mayor que 3. ¿cuál
  es el nuevo espacio muestra?¿Cuál la probabilidad de \(P(B\vert A)\)?
\item
  Un candado de seguridad se puede abrir ingresando 2 dígitos (cada uno
  de \(1, 2,\ldots, 9\)) que tendrá \(10^{2} = 100\) códigos posibles.
  Un viajero ha olvidado el código e intenta encontrar el código
  eligiendo 2 dígitos al azar. Si el código no abre el candado, el
  viajero intenta con otro código de los pares. ¿Cuál es la probabilidad
  de que el nuevo código abre el caso, o cualquier intento posterior?
\item
  Sea \(A\) y \(B\) eventos independientes con \(P(A)=1/4\) y
  \(P(B)=2/3\). Calcule las siguientes probabilidades: (a)
  \(P(A\cap B)\), (b)\(P(A\cap B^{c})\), (c)\(P(A^{c}\cap B^{c})\), (d)
  \(P(A^{c}\cap B)\), (e) \(P((A\cap B)^{c})\)
\item
  Para tres eventos \(A\), \(B\) y \(C\) de muestre que
\end{itemize}

\[
P(A\cap B \vert C)= P(A\vert B \cap C)P(B\vert C)
\]

donde \(P(C)>0\)

    \hypertarget{ley-de-probabilidad-total-o-teorema-de-particiuxf3n}{%
\subsection{Ley de probabilidad total o teorema de
partición}\label{ley-de-probabilidad-total-o-teorema-de-particiuxf3n}}

Supongamos que \(A_{1},A_{2},\ldots, A_{k}\) representa una partición de
S en k conjuntos mutamente exculyentes. Cuando se lleva a cabo un
experimento aleatorio, solo uno de los eventos puede tener lugar.

Suponga que \(B\) es otro evento asociado con el mismo experimento
aleatorio. Entonces \(B\) debe estar formado por la suma de las
intersecciones de \(B\) con eventos en la partición. Algunos de estos
estarán vacíos, pero esto no importa. Podemos decir eso \(B\) es la
unión de las intersecciones de B con cada \(A_{i}\). Así

\[
B=\bigcup\limits_{i=1}^{k}B \cap A_{i}
\]

el punto significativo es que cualquier par de estos eventos es
mutuamente excluyente. Esto resulta en

\[
P(B)=\displaystyle \sum\limits_{i=1}^{k}P(B \cap A_{i})
\]

Entonces la ecuación

\[
P(B\cap A_{i})= P(B\vert A_{i}) P(A_{i})
\]

Entonces también puede expresarse como

\[
P(B)=\displaystyle \sum\limits_{i=1}^{k}P(B\vert A_{i}) P(A_{i})
\]

La cual es la ley de probabilidad total o teorema de partición

\textbf{Regla de Bayes}

Regresemos al ejemplo de la tabla, en el que se selecciona un individuo
al azar de entre los adultos de una pequeña ciudad. Suponga que ahora se
nos da la información adicional de que 36 de los empleados y 12 de los
desempleados son miembros de un Club . Deseamos encontrar la
probabilidad del evento \(A\) de que el individuo seleccionado sea
miembro del Club. Podemos remitirnos a la figura y escribir \(A\) como
la unión de los dos eventos mutuamente excluyentes \(E \cap A\) y
\(E' \cap A\). Por lo tanto, \(A = (E \cap A) \cup (E' \cap A)\), y
mediante el los teoremas anteriores, podemos escribir

\[
P(A) = P [(E \cap A) \cup (E' \cap A)] = P(E \cap A) + P(E' \cap A) = P(E)P(A\vert E) + P(E')P(A\vert E').
\]

con los datos adicionales antes dados para el conjunto \(A\), nos
permiten calcular

\[
P (E) = 600/900= 2/3 \quad P (A \vert E ) =36/600= 3/50
\]

\[
P (E') = 1/3 \quad P (A \vert E' ) =12/300= 1/25
\]

con esta información se puede generar un diagrama de árbol

Si mostramos estas probabilidades mediante el diagrama de árbol, donde
la primera rama da la probabilidad \(P(E)P(A\vert E)\) y la segunda rama
da la probabilidad P(E')P(A\textbar E'), deducimos que

\[
P(A)= \frac{2}{3}\frac{3}{50}+ \frac{1}{3} \frac{1}{25}= \frac{4}{75}
\]

\textbf{Regla general de Bayes}

Si los enventos \(B_{1},B_{2},\ldots,B_{k}\) constituyen una partición
del espacio muestral S, donde \(P(B_{i}) \neq 0\) para
\(i=1,2,\ldots, k\) entonces, para cualquier evento A en S, tal que
\(P(A)\neq 0\),

\[
    P(B_{r}\vert A)= \frac{P(B_{r}\cap A)}{\displaystyle \sum\limits_{i=1}^{k} P(B_{i}\cap A)}= \frac{P(B_{r})P(A\vert B_{r})}{ \displaystyle \sum\limits_{i=1}^{k}P(B_{i})P(A\vert B_{i})}
\]

\emph{Ejercicios}

\begin{itemize}
\item
  Tres máquinas de cierta planta de ensamble, \(B_{1} , B_{2}\) y
  \(B_{3}\) , montan 30\%, 45\% y 25\% de los productos,
  respectivamente. Se sabe por experiencia que 2\%, 3\% y 2\% de los
  productos ensamblados por cada máquina, respectivamente, tienen
  defectos. Ahora bien, suponga que se selecciona de forma aleatoria un
  producto terminado. ¿Cuál es la probabilidad de que esté defectuoso?
\item
  Con referencia al ejemplo anterior, si se elige al azar un producto y
  se encuentra que está defectuoso, ¿cuál es la probabilidad de que haya
  sido ensamblado con la máquina B 3 ?
\item
  Una empresa de manufactura emplea tres planos analíticos para el
  diseño y desarrollo de un producto específi co. Por razones de costos
  los tres se utilizan en momentos diferentes. De hecho, los planos 1, 2
  y 3 se utilizan para 30\%, 20\% y 50\% de los productos,
  respectivamente. La tasa de defectos difi ere en los tres
  procedimientos de la siguiente manera, \[
  P(D\vert P_{1} ) = 0.01, P(D\vert P_{2} ) = 0.03, P(D \vert P_{3} ) = 0.02,
  \] en donde \(P(D\vert P_{j} )\) es la probabilidad de que un producto
  esté defectuoso, dado el plano \(j\). Si se observa un producto al
  azar y se descubre que está defectuoso, ¿cuál de los planos tiene más
  probabilidades de haberse utilizado y, por lo tanto, de ser el
  responsable?
\end{itemize}

    \hypertarget{variables-aleatorias-discretas}{%
\subsection{Variables aleatorias
discretas}\label{variables-aleatorias-discretas}}

Una \textbf{variable aleatoria} es una función que asocia un número real
con cada elemento del espacio muestral. Utilizaremos una letra
mayúscula, digamos \(X\), para denotar una variable aleatoria, y su
correspondiente letra minúscula, \(x\) en este caso, para uno de sus
valores.

Ejemplo. cuando se prueban tres componentes electrónicos, el espacio
muestral que ofrece una descripción detallada de cada posible resultado
se escribe como

\[
S = \lbrace NNN, NND, NDN, DNN, NDD, DND, DDN, DDD\rbrace,
\]

donde \(N\) denota ``no defectuoso'', y \(D\), ``defectuoso''. Es
evidente que nos interesa el número de componentes defectuosos que se
presenten. De esta forma, a cada punto en el espacio muestral se le
asignará un valor numérico de 0, 1, 2 o 3.

\emph{Ejericios}

\begin{itemize}
\item
  De una urna que contiene 4 bolas rojas y 3 negras se sacan 2 bolas de
  manera sucesiva, sin reemplazo. ¿Cuáles son Los posibles resultados y
  los valores y de la variable aleatoria \(Y\)? si \(Y\) es el número de
  bolas rojas.
\item
  El empleado de un almacén regresa tres cascos de seguridad al azar a
  tres obreros de un taller siderúrgico que ya los habían probado. Si
  Smith, Jones y Brown, en ese orden, reciben uno de los tres cascos,
  liste los puntos muestrales para los posibles órdenes en que el
  empleado del almacén regresa los cascos, después calcule el valor m de
  la variable aleatoria M que representa el número de emparejamientos
  correctos.
\end{itemize}

Cada variables aleatorias tiene, a su vez, espacios muestrales cuyos
elementos generalmente se denotan con letras minúsculas como
\(x_{1}, x_{2}, x_{3},\ldots\) para la variable aleatoria \(X\). Ahora
estamos interesados en asignar probabilidades a eventos como
\(P (X = x_{1})\), la probabilidad de que la variable aleatoria \(X\)
sea \(x_{1}\), o \(P (X \leq x_{2})\), la probabilidad de que la
variable aleatoria variable es menor o igual que \(x_{2}\)

Si el espacio muestral es finito o contablemente infinito en los números
enteros (es decir, los elementos \(x_{1}, x_{2}, x_{3},\ldots\) o pueden
contarse con números enteros, digamos \(0, 1, 2,\ldots\)) entonces
\textbf{la variable aleatoria es discreta}. Técnicamente, el conjunto
\(\lbrace x_{i}\rbrace\) será un subconjunto contable \(V\), digamos, de
los números reales \(R\). Podemos representar el
\(\lbrace x_{i}\rbrace\) genéricamente por la variable \(x\) con
\(x \in V\). Por ejemplo, \(V\) podría ser el conjunto

\[
\lbrace 0, 1/2 , 1, 3/2 , 2, 5/2 , 3, \ldots \rbrace 
\]

En muchos casos \(V\) consiste simplemente de enteros o un subconjunto
de enteros, tales

\[
V = \lbrace0, 1 \rbrace \text{  o  } V = \lbrace 0, 1, 2, 3, \ldots \rbrace  \text{ incluso } V = \lbrace \ldots − 3, −2, −1, 0, 1, 2, 3, \ldots \rbrace
\]

La probabilidad se denota por \[
p(x_{i} ) = P(X = x_{i} )
\]

    \hypertarget{distribuciuxf3n-de-probabilidad}{%
\subsection{Distribución de
probabilidad}\label{distribuciuxf3n-de-probabilidad}}

Una variable aleatoria discreta toma cada uno de sus valores con cierta
probabilidad. Por ejemplo, si los pesos son iguales para los eventos del
ejemplo de los cascos, la probabilidad de que el obrero no reciba el
casco correcto M toma m valores posibles y sus probabilidades son

\begin{longtable}[]{@{}cccc@{}}
\toprule
m & 0 & 1 & 3\tabularnewline
\midrule
\endhead
P( M = m) & 1/3 & 1/2 & 1/6\tabularnewline
\bottomrule
\end{longtable}

Con frecuencia es conveniente representar todas las probabilidades de
una variable aleatoria \(X\) usando una fórmula, la cual necesariamente
sería una función de los valores numéricos \(x\) que denotaremos con
\(p (x)\), \(f(x)\), \(g (x)\) y así sucesivamente. Por lo tanto,
escribimos \(p(x_{i}) = P(X = x_{i});\) es decir, \(p (3) = P(X = 3)\).
Al conjunto de pares ordenados \((x_{i},p (x_{i}))\) se le llama
\textbf{función de probabilidad, función de masa de probabilidad o
distribución de probabilidad de la variable aleatoria discreta \(X\)}.
Ya que los valores \(x\) son mutamente excluyentes y exhaustivos
entonces cumplen

\[
0 \leq p (x_{i}) \leq 1 \quad \forall i,
\]

\[
\sum\limits_{i=1}^{\infty} p (x_{i}) =1   
\]

\[
P(X \leq x_{k})= \sum\limits_{i=1}^{k} p (x_{i}) \text{ para  } k=0,1,2,\ldots   
\]

\emph{Ejemplos}

\begin{itemize}
\tightlist
\item
  Un embarque de 20 computadoras portátiles similares para una tienda
  minorista contiene 3 que están defectuosas. Si una escuela compra al
  azar 2 de estas computadoras, calcule la distribución de probabilidad
  para el número de computadoras defectuosas.
\end{itemize}

Respuesta: sea X una variable aleatoria cuyos valores x son los números
posibles de computadoras defectuosas compradas por la escuela. Entonces
x sólo puede asumir los números 0, 1 y 2. Así,

\[
f(0)=P(X=0)= \frac{\binom{3}{0} \binom{17}{2} }{\binom{20}{2}}=\frac{68}{95}
\]

\[
f(1)=P(X=1)= \frac{\binom{3}{1} \binom{17}{1} }{\binom{20}{2}}=\frac{51}{190}
\]

\[
f(2)=P(X=2)= \frac{\binom{3}{2} \binom{17}{0} }{\binom{20}{2}}=\frac{3}{190}
\]

por consiguiente

\begin{longtable}[]{@{}cccc@{}}
\toprule
\(x\) & 0 & 1 & 2\tabularnewline
\midrule
\endhead
\(f(x)\) & \(\frac{68}{95}\) & \(\frac{51}{190}\) &
\(\frac{3}{190}\)\tabularnewline
\bottomrule
\end{longtable}

\begin{itemize}
\tightlist
\item
  Se lanza un dado justo hasta que el primer 6 aparezcan boca arriba.
  Encuentre la probabilidad de que el primer 6 aparezcan en el n-ésimo
  lanzamiento.
\end{itemize}

Respuesta: Sea la variable aleatoria \(N\) el número de lanzamientos
hasta que el primer 6 aparezcan boca arriba. Este es un ejemplo de una
variable aleatoria discreta \(N\) con un número infinito de posibles
resultados

\[
\lbrace 1, 2, 3, \ldots \rbrace 
\]

La probabilidad de que aparezca un \(6\) para cualquier lanzamiento es
\(1/6\) y de que aparezca cualquier otro número es \(5/6\). Por lo
tanto, la probabilidad de que aparezcan \(n - 1\) números distintos del
\(6\) seguido de un 6 es

\[
P(N=n)= \binom{5}{6}^{n-1} \binom{1}{6}= \frac{5^{n-1}}{6^{n}}
\]

que es la función de masa de probabilidad para esta variable aleatoria.

En este ejemplo, la distribución tiene la probabilidad

\[
P(N \leq k) = \frac{1}{6}+ \frac{1}{6}\frac{5}{6} + \cdots \frac{1}{6}\left(\frac{5}{6}\right)^{k-1} = \frac{1}{6} \displaystyle \sum_{i=1}^{k}{\left(\frac{5}{6}\right)}^{i-1}= 1- \left(\frac{5}{6}\right)^{k}
\]

para \(k = 1, 2, \ldots 6\) after summing the geometric series.

\emph{Ejercicio} - Si una agencia automotriz vende 50\% de su inventario
de cierto vehículo extranjero equipado con bolsas de aire laterales,
calcule una fórmula para la distribución de proba bilidad del número de
automóviles con bolsas de aire laterales entre los siguientes 4
vehículos que venda la agencia.

\hypertarget{funciuxf3n-de-la-distribuciuxf3n-acumulativa}{%
\subsection{Función de la distribución
acumulativa}\label{funciuxf3n-de-la-distribuciuxf3n-acumulativa}}

Existen muchos problemas en los que desearíamos calcular la probabilidad
de que el valor observado de una variable aleatoria \(X\) sea menor o
igual que algún número real \(x\). Al escribir \(F(x) = P(X \leq x)\)
para cualquier número real \(x\), definimos \(F(x)\) como la función de
la distribución acumulativa de la variable aleatoria \(X\).

La función de la distribución acumulativa \(F(x)\) de una variable
aleatoria discreta \(X\) con distribución de probabilidad \(f(x)\) es

\[
F(x ) = P (X\leq x ) = \sum_{t\leq x} f (t), \text{  para  } -\infty < x < \infty.
\]

\emph{Ejercicio}

Realice un tabla de distribución acumulativa del ejemplo de
emparejamiento de los cascos.

\hypertarget{variables-y-distribuciones-de-probabilidad-continua}{%
\subsection{Variables y distribuciones de probabilidad
continua}\label{variables-y-distribuciones-de-probabilidad-continua}}

En muchas aplicaciones, la variable aleatoria discreta es inapropiada
para problemas en los que puede tomar cualquier valor real en un
intervalo. Por ejemplo, la variable aleatoria \(T\) podría ser el tiempo
medido desde el momento \(t = 0\) hasta que falla una bombilla. Este
podría ser cualquier valor \(t \leq 0\). En este caso, \(T\) se denomina
variable aleatoria continua. Generalmente, si X es una variable
aleatoria continua, existen dificultades matemáticas para definir el
evento X = x: la probabilidad generalmente se define como cero. Las
probabilidades de las variables aleatorias continuas solo se pueden
definir en intervalos de valores como, por ejemplo, en
\(P (x_{1} <X <x_{2})\).

Definimos una función de densidad de probabilidad (fdp o pdf) \(f(x\))
en el internbalo \(-\infty < x < \infty\) la cual tiene las siguientes
propiedades

\(f(x)\leq 0, \quad (-\infty < x < \infty)\)

\(P(x_{1}\leq x_{2} ) = \int_{x_{1}}^{x_{2}}f(x)dx , \quad \text{tal que} \quad -\infty < x_{1}< x_{2}< \infty\)

\(\int_{-\infty}^{\infty}f(x)dx=1\)

Posible grafica de función de densidad \(f(x)\) se muestra a
continuación

La función de distribución acumulada (CDF) F(x) de que \(X\) sea igual o
menor a \(x\) es

\[
F(x)= P(X\leq x)= \int_{-\infty}^{x} f(u)du
\] y

\[
P(x_{1}\leq x \leq x_{2})= \int_{x_{1}}^{x_{2}} f(u)du= F(x_{2})-F(x_{1})
\]

Ejemplo de un función CDF

\emph{Ejercicios}

\begin{itemize}
\tightlist
\item
  Suponga que el error en la temperatura de reacción, en °C, en un
  experimento de laboratorio controlado, es una variable aleatoria
  continua X que tiene la función de densidad de probabilidad
\end{itemize}

\begin{equation}
     \label{eq:aqui}
     f(x) = \left\{
           \begin{array}{ll}
             \frac{x^2}{3}    & \mathrm{si\ } -1 \le x \le 2 \\
             0 & \mathrm{\text{otro caso}  } 
           \end{array}
         \right.
   \end{equation}

\begin{enumerate}
\def\labelenumi{\alph{enumi})}
\item
  Verifique que \(f (x)\) es una función de densidad.
\item
  Calcule \(P(0 < X \leq 1)\).
\end{enumerate}

\begin{itemize}
\item
  Calcule la acumulada \(F(x)\) para la función de densidad del
  ejercicio anterior y utilice el resultado para evaluar
  \(P(0 < X \leq 1)\).
\item
  Demuestre que
\end{itemize}

\begin{equation}
     \label{eq:aqui-le-mostramos-como-hacerle-la-llave-grande}
     f(x) = \left\{
           \begin{array}{ll}
             1/(b-a)     & \mathrm{si\ } a \leq x \le b \\
             0 & \mathrm{\text{otro caso}  } 
           \end{array}
         \right.
   \end{equation}

es una posible función de densidad de probabilidad. Encuentre su función
acumulada.

    \hypertarget{esperanza-matemuxe1tica}{%
\subsection{Esperanza matemática}\label{esperanza-matemuxe1tica}}

El valor promedio se le conoce como media de la variable aleatoria \(X\)
o media de la distribución de probabilidad de \(X\), y se le denota como
\$ \mu\_\{x\} \$ o simplemente como \(\mu\) cuando es evidente a qué
variable aleatoria se está haciendo referencia. También es común entre
los estadísticos referirse a esta media como la esperanza matemática o
el valor esperado de la variable aleatoria \(X\) y y denotarla como
\(E(x)\)

\$ \mu=E(X)= \sum\emph{\{i=0\}\^{}\{\infty\}\{x}\{i\}p(x\_\{i\})\}
\quad \text{Variable aleatoria discreta}\$

\(\mu=E(X)= \int_{-\infty}^{\infty}{xf(x)dx} \quad \text{Variable aleatoria continua}\)

donde \(f(x)\) es la función de densidad de probabilidad. Puede
interpretarse como el promedio ponderado de los valores de \(X\) en su
espacio muestral, donde los pesos son la función de probabilidad o la
función de densidad.

\hypertarget{varianza-de-variables-aleatorias}{%
\subsection{Varianza de variables
aleatorias}\label{varianza-de-variables-aleatorias}}

Una medida que se usa, además de la media, es la \textbf{varianza} de
\(X\) denotada por \(V(X)\) o \(\sigma^{2}\). Esto da una medida de
variación o dispersión de la distribución de probabilidad de \(X\), y se
define por

\[
\sigma^{2}=V(x)= E[(X-E(X))^2]= E[(X-\mu)^2]
\]

\begin{equation}
     \label{eq:aqui2}
     \sigma^{2}= \left\{
           \begin{array}{ll}
             \displaystyle \sum_{i=0}^{\infty}{(x_{i}-\mu)^2 p(x_{i})}    &  \text{si es discreta}\\
           \int_{-\infty}^{\infty}{(x-\mu)^2 p(x)dx}   & \text{si es continua} 
    \end{array}
         \right.
 \end{equation}

Una función de una variable aleatoria es en sí misma una variable
aleatoria. Si \(h (X)\) es una función de la variable aleatoria \(X\),
entonces se puede demostrar que la expectativa de \(h (X)\) está dada
por

\begin{equation}
     \label{eq:aqui3}
     E[h(X)]= \left\{
           \begin{array}{ll}
             \displaystyle \sum_{i=0}^{\infty}{h(x_{i}) p(x_{i})}    &  \text{si es discreta}\\
           \int_{-\infty}^{\infty}{ h(x)p(x)dx}   & \text{si es continua} 
           \end{array}
         \right.
 \end{equation}

Es relativamente sencillo derivar los siguientes resultados para la
expectativa y la varianza de una función lineal de X: \[
E(aX + b) = aE(X) + b = a\mu + b,
\] \[
V(aX + b) = E[(aX + b − a\mu − b)^2 ]
= E[(aX − a\mu)^2 ] = a^2 E[(X − \mu)^2 ] = a^2 V(X)
\]

donde \(a\) y \(b\) son constantes. Tenga en cuenta que la traslación de
la variable aleatoria no afecta la varianza. También \[
V(X) = E[(X − \mu)^2 ] = E(X^2 ) − 2\mu E(X) +\mu^2 = E(X^2 ) − \mu^2 
\]

la cual se conoce como la fórmula computacional para la varianza.

Para las expectativas, se puede demostrar de manera más general que \[
E\left[  \displaystyle \sum_{i=0}^{k}{a_{i}h_{i}(X)}
 \right]=  \displaystyle \sum_{i=0}^{k}{a_{i}E[h_{i}(X)]}
\]

donde \(a_{i}, i = 1, 2, \ldots\) , \(k\) son constantes y
\(h_{i} (X), i = 1, 2, \ldots\), \(k\) son funciones de la variable
aletoria \(X\).

\emph{Ejercicios} - Un inspector de calidad obtiene una muestra de un
lote que contiene 7 componentes; el lote contiene 4 componentes buenos y
3 defectuosos. El inspector toma una muestra de 3 componentes. Calcule
el valor esperado del número de componentes buenos en esta muestra.

\begin{itemize}
\item
  Cierto día un vendedor de una empresa de aparatos médicos tiene dos
  citas. Considera que en la primera cita tiene 70 por ciento de
  probabilidades de cerrar una venta, por la cual podría obtener una
  comisión de 1000. Por otro lado, cree que en la segunda cita sólo
  tiene 40 por ciento de probabilidades de cerrar el trato, del cual
  obtendría 1500 de comisión. ¿Cuál es su comisión esperada con base en
  dichas probabilidades? Suponga que los resultados de las citas son
  independientes.
\item
  Sea \(X\) la variable aleatoria que denota la vida en horas de cierto
  dispositivo electrónico. La función de densidad de probabilidad es
\end{itemize}

\begin{equation}
     \label{eq:aqui4}
     f(x)= \left\{
           \begin{array}{ll}
             \frac{20000}{x^3}  &  x >100 \\
           0   & \text{en otro caso} 
           \end{array}
         \right.
 \end{equation} calcule la vida esperada para esta clase de dispositivo.

    \hypertarget{algunas-distribuciones-de-probabilidad-discreta}{%
\section{Algunas distribuciones de probabilidad
discreta}\label{algunas-distribuciones-de-probabilidad-discreta}}

A menudo las observaciones que se generan mediante diferentes
experimentos estadísticos tienen el mismo tipo general de
comportamiento. En consecuencia, las variables aleatorias discretas
asociadas con estos experimentos se pueden describir esencialmente con
la misma distribución de probabilidad y, por lo tanto, es posible
representarlas usando una sola fórmula. De hecho, se necesitan sólo unas
cuantas distribuciones de probabilidad importantes.

\hypertarget{distribuciones-binomial-y-multinomial}{%
\subsection{Distribuciones binomial y
multinomial}\label{distribuciones-binomial-y-multinomial}}

Con frecuencia un experimento consta de pruebas repetidas, cada una con
dos resultados posibles que se pueden denominar éxito o fracaso. El
proceso se conoce como proceso de Bernoulli y cada ensayo se denomina
experimento de Bernoulli.

\emph{El proceso de Bernoulli}

Solo hay dos resultados: un ``éxito'' \((X = 1)\) con probabilidad \(p\)
o un ``fracaso'' \((X = 0)\) con probabilidad \(q = 1 - p\). El valor de
la variable aleatoria \(X\) se utiliza como indicador del resultado. Por
ejemplo, en un solo lanzamiento de moneda, X = 1 está asociado con la
aparición, o la presencia de la característica, de una cara, y X = 0 con
una cruz, o la ausencia de una cara.

La función de probabilidad de esta variable aleatoria se puede expresar
como

\[
p_{1} = P(X = 1) = p, p_{0} = P(X = 0) = q,
\]

El valor esperado de \(X\) es

\[
\mu = E(x)= (0)(q)+ (1)(p)=p,
\]

Y la varianza
\[                                                        \sigma^2 = V(X) = E(X^2 ) − \mu^2 = (0^2)(q) + (1^2)(p) − p^2 = pq.                                               \]

Supongamos ahora que estamos interesados en variables aleatorias
asociadas con repeticiones independientes de experimentos de Bernoulli,
cada una con una probabilidad de éxito, \(p\). Considere primero la
distribución de probabilidad de una variable aleatoria \(X\) que es el
número de éxitos en un número fijo de ensayos independientes, \(n\). Si
hay \(k\) éxitos y \(n - k\) fracasos en \(n\) ensayos, entonces cada
secuencia de \(1\) y \(0\) tiene la probabilidad
\(P (X = k) = p^k q^{n − k}\). El número de formas en que se pueden
organizar \(x\) éxitos en \(n\) ensayos es la expresión binomial

\[
\frac{n!}{k!(n-k)!} \quad \text{también se expresa con} \quad \binom{n}{k}
\]

Dado que cada una de estas secuencias mutuamente excluyentes ocurre con
probabilidad \(p^k q^{n − k}\), la función de probabilidad de esta
variable aleatoria viene dada por

\[
b(k; n, p) = p_{k}=\binom{n}{k}p^k q^{n − k} \quad k=0,1,2,\ldots,n
\]

Se le conoce como binomial porque el término \((n+1)\) tiene
correspondencia con la expansión del binomio \((p+q)^2\)

\[
\displaystyle \sum_{k=0}^{n}{p_{k}}= \sum_{k=0}^{n} \binom{n}{k}p^k q^{n − k}= (p+q)^n=1 
\]

La media y la varianza son \(np\) y \(npq\) respectivamente, la cual es
n veces la de Bernoulli.

Con frecuencia nos interesamos en problemas donde se necesita obtener
\(P(X < r) \text{ o } P(a \leq X \leq b)\). Estas se obtienen con
sumatorias binomiales

\[
P(X < r)= B (r; n, p) =\displaystyle \sum_{x=0}^{r} b(x; n, p)
\]

\emph{Ejercicios}

\begin{enumerate}
\def\labelenumi{\arabic{enumi}.}
\item
  Considere el conjunto de experimentos de Bernoulli en el que se
  seleccionan tres artículos al azar de un proceso de producción, luego
  se inspeccionan y se clasifican como defectuosos o no defectuosos. Un
  artículo defectuoso se designa como un éxito. El número de éxitos es
  una variable aleatoria \(X\) que toma valores integrales de cero a 3.
  Considere que la probabilidad de defecto es 0.25. Encuentre los ocho
  resultados posibles, los valores correspondientes de X, tabule la
  función de distribución y calcule la probabilidad de que sean
  exactamente dos defectuosos usando la función de distribución.
\item
  La probabilidad de que cierta clase de componente sobreviva a una
  prueba de choque es de 3/4. Calcule la probabilidad de que sobrevivan
  exactamente 2 de los siguientes 4 componentes que se prueben.
\item
  La probabilidad de que un paciente se recupere de una rara enfermedad
  sanguínea es de 0.4. Si se sabe que 15 personas contrajeron la
  enfermedad, ¿cuál es la probabilidad de que a) sobrevivan al menos 10,
  b) sobrevivan de 3 a 8, y c) sobrevivan exactamente 5?
\item
  Una cadena grande de tiendas al detalle le compra cierto tipo de
  dispositivo electrónico a un fabricante, el cual le indica que la tasa
  de dispositivos defectuosos es de 3\%.

  \begin{enumerate}
  \def\labelenumii{\alph{enumii})}
  \tightlist
  \item
    El inspector de la cadena elige 20 artículos al azar de un
    cargamento. ¿Cuál es la probabilidad de que haya al menos un
    artículo defectuoso entre estos 20?
  \item
    Suponga que el detallista recibe 10 cargamentos en un mes y que el
    inspector prueba aleatoriamente 20 dispositivos por cargamento.
    ¿Cuál es la probabilidad de que haya exactamente tres cargamentos
    que contengan al menos un dispositivo defectuoso de entre los 20
    seleccionados y probados?
  \end{enumerate}
\item
  Se conjetura que hay impurezas en 30\% del total de pozos de agua
  potable de cierta co- munidad rural. Para obtener información sobre la
  verdadera magnitud del problema se determina que debe realizarse algún
  tipo de prueba. Como es muy costoso probar todos los pozos del área,
  se eligen 10 al azar para someterlos a la prueba.

  \begin{enumerate}
  \def\labelenumii{\alph{enumii})}
  \tightlist
  \item
    Si se utiliza la distribución binomial, ¿cuál es la probabilidad de
    que exactamente 3 pozos tengan impurezas, considerando que la
    conjetura es correcta?
  \item
    ¿Cuál es la probabilidad de que más de 3 pozos tengan impurezas?
  \end{enumerate}
\item
  Calcule la media y la varianza de la variable aleatoria binomial del
  ejemplo 3
\end{enumerate}

    \begin{tcolorbox}[breakable, size=fbox, boxrule=1pt, pad at break*=1mm,colback=cellbackground, colframe=cellborder]
\prompt{In}{incolor}{29}{\boxspacing}
\begin{Verbatim}[commandchars=\\\{\}]
\PY{c+c1}{\PYZsh{} A Python program to print all  }
\PY{c+c1}{\PYZsh{} combinations of given length }
\PY{k+kn}{from} \PY{n+nn}{itertools} \PY{k+kn}{import} \PY{n}{combinations} 
\PY{k+kn}{import} \PY{n+nn}{numpy} \PY{k}{as} \PY{n+nn}{np}  
\PY{k+kn}{from} \PY{n+nn}{math} \PY{k+kn}{import} \PY{n}{comb}
\PY{c+c1}{\PYZsh{} Get all combinations of [1, 2, 3] }
\PY{c+c1}{\PYZsh{} and length 2 }
\PY{n}{combi} \PY{o}{=} \PY{n}{combinations}\PY{p}{(}\PY{p}{[}\PY{l+m+mi}{0}\PY{p}{,}\PY{l+m+mi}{1}\PY{p}{,} \PY{l+m+mi}{2}\PY{p}{,} \PY{l+m+mi}{3}\PY{p}{,}\PY{l+m+mi}{4}\PY{p}{,}\PY{l+m+mi}{5}\PY{p}{,}\PY{l+m+mi}{6}\PY{p}{]}\PY{p}{,} \PY{l+m+mi}{3}\PY{p}{)} 
\PY{n}{r}\PY{o}{=}\PY{l+m+mi}{0}\PY{p}{;}
\PY{c+c1}{\PYZsh{} Print the obtained combinations }
\PY{c+c1}{\PYZsh{} for i in list(combi): }
\PY{c+c1}{\PYZsh{}     print (i)}
\PY{c+c1}{\PYZsh{}     r=r+1;}
\PY{c+c1}{\PYZsh{} print(r)}
\PY{c+c1}{\PYZsh{} l=[10,2,8,9,1,8,6,5,7]}
\PY{c+c1}{\PYZsh{} print(np.mean(l))}
\PY{c+c1}{\PYZsh{} n=len(l)}
\PY{c+c1}{\PYZsh{} print(n*np.mean(l))}
\PY{c+c1}{\PYZsh{} print(np.sum(l))}

\PY{c+c1}{\PYZsh{}comb(10,2)*(0.3)**2*(0.7)**8}
\PY{n+nb}{print}\PY{p}{(}\PY{n}{comb}\PY{p}{(}\PY{l+m+mi}{10}\PY{p}{,}\PY{l+m+mi}{0}\PY{p}{)}\PY{o}{*}\PY{p}{(}\PY{l+m+mf}{0.3}\PY{p}{)}\PY{o}{*}\PY{o}{*}\PY{l+m+mi}{0}\PY{o}{*}\PY{p}{(}\PY{l+m+mf}{0.7}\PY{p}{)}\PY{o}{*}\PY{o}{*}\PY{l+m+mi}{10}\PY{p}{)}
\PY{n+nb}{print}\PY{p}{(}\PY{n}{comb}\PY{p}{(}\PY{l+m+mi}{10}\PY{p}{,}\PY{l+m+mi}{1}\PY{p}{)}\PY{o}{*}\PY{p}{(}\PY{l+m+mf}{0.3}\PY{p}{)}\PY{o}{*}\PY{o}{*}\PY{l+m+mi}{1}\PY{o}{*}\PY{p}{(}\PY{l+m+mf}{0.7}\PY{p}{)}\PY{o}{*}\PY{o}{*}\PY{l+m+mi}{9}\PY{p}{)}
\PY{n+nb}{print}\PY{p}{(}\PY{n}{comb}\PY{p}{(}\PY{l+m+mi}{10}\PY{p}{,}\PY{l+m+mi}{2}\PY{p}{)}\PY{o}{*}\PY{p}{(}\PY{l+m+mf}{0.3}\PY{p}{)}\PY{o}{*}\PY{o}{*}\PY{l+m+mi}{2}\PY{o}{*}\PY{p}{(}\PY{l+m+mf}{0.7}\PY{p}{)}\PY{o}{*}\PY{o}{*}\PY{l+m+mi}{8}\PY{p}{)}
\PY{n+nb}{print}\PY{p}{(}\PY{n}{comb}\PY{p}{(}\PY{l+m+mi}{10}\PY{p}{,}\PY{l+m+mi}{3}\PY{p}{)}\PY{o}{*}\PY{p}{(}\PY{l+m+mf}{0.3}\PY{p}{)}\PY{o}{*}\PY{o}{*}\PY{l+m+mi}{3}\PY{o}{*}\PY{p}{(}\PY{l+m+mf}{0.7}\PY{p}{)}\PY{o}{*}\PY{o}{*}\PY{l+m+mi}{7}\PY{p}{)}

\PY{c+c1}{\PYZsh{}print(1\PYZhy{}0.2334\PYZhy{}0.02824\PYZhy{}0.121\PYZhy{}0.26682)}
\PY{n+nb}{print}\PY{p}{(}\PY{l+m+mi}{15}\PY{o}{*}\PY{l+m+mf}{0.4}\PY{o}{*}\PY{l+m+mf}{0.6}\PY{p}{)}
\end{Verbatim}
\end{tcolorbox}

    \begin{Verbatim}[commandchars=\\\{\}]
0.028247524899999984
0.12106082099999993
0.23347444049999988
0.2668279319999998
3.5999999999999996
    \end{Verbatim}

    \begin{tcolorbox}[breakable, size=fbox, boxrule=1pt, pad at break*=1mm,colback=cellbackground, colframe=cellborder]
\prompt{In}{incolor}{ }{\boxspacing}
\begin{Verbatim}[commandchars=\\\{\}]

\end{Verbatim}
\end{tcolorbox}

    \begin{tcolorbox}[breakable, size=fbox, boxrule=1pt, pad at break*=1mm,colback=cellbackground, colframe=cellborder]
\prompt{In}{incolor}{ }{\boxspacing}
\begin{Verbatim}[commandchars=\\\{\}]

\end{Verbatim}
\end{tcolorbox}


    % Add a bibliography block to the postdoc
    
    
    
\end{document}
